\documentclass[11pt]{article}

\usepackage{
    amssymb,
    amsmath,
    amsfonts,
    calc,
    eurosym,
    geometry,
    ulem,
    graphicx,
    caption,
    color,
    setspace,
    sectsty,
    comment,
    footmisc,
    caption,
    % natbib,
    pdflscape,
    subcaption,
    subfiles,
    titling,
    array,
    hyperref,
    booktabs,
    longtable,
    float,
    authblk,
    makecell,
    threeparttable,
    pgfplots,
    minitoc,
    tocbasic}

\usepackage[page]{appendix} % print appendices title

\usepackage[
    backend=biber,
    style=nature,
    date=year,
    doi=true,
    isbn=false,
    url=false,
    eprint=false
]{biblatex}

% Make the "Part I" text invisible
\renewcommand \thepart{}
\renewcommand \partname{}
\DeclareTOCStyleEntry[dynnumwidth=true,
                      numsep=1em,
]{tocline}{section}
\DeclareTOCStyleEntry[dynnumwidth=true,
                      numsep=1em,
]{tocline}{subsection}
\AtEveryBibitem{%
  \clearfield{note}%
}
\AtEveryCitekey{\clearlist{publisher}}
\AtEveryBibitem{\clearlist{publisher}}

\usepackage{pgf,tikz}
\usetikzlibrary{arrows, automata}
\usetikzlibrary{shapes.geometric}
\usetikzlibrary{positioning,calc, decorations.pathreplacing}

\usepackage{siunitx}
\newcolumntype{d}{S[input-symbols = ()]}

\normalem

\renewcommand\Affilfont{\small\itshape}

\onehalfspacing
\newtheorem{theorem}{Theorem}
\newtheorem{corollary}[theorem]{Corollary}
\newtheorem{proposition}{Proposition}
\newenvironment{proof}[1][Proof]{\noindent\textbf{#1.} }{\ \rule{0.5em}{0.5em}}

\newtheorem{hyp}{Hypothesis}
\newtheorem{subhyp}{Hypothesis}[hyp]
\renewcommand{\thesubhyp}{\thehyp\alph{subhyp}}

\newcommand{\red}[1]{{\color{red} #1}}
\newcommand{\blue}[1]{{\color{blue} #1}}

\newcolumntype{L}[1]{>{\raggedright\arraybackslash}m{#1}}
\newcolumntype{C}[1]{>{\centering\arraybackslash}m{#1}}
\newcolumntype{R}[1]{>{\raggedleft\arraybackslash}m{#1}}
\subsubsectionfont{\normalfont\itshape}

\usepackage{mathtools}

\usepackage{letltxmacro}
\LetLtxMacro\orgvdots\vdots
\LetLtxMacro\orgddots\ddots

\makeatletter
\DeclareRobustCommand\vdots{%
  \mathpalette\@vdots{}%
}
\newcommand*{\@vdots}[2]{%
  % #1: math style
  % #2: unused
  \sbox0{$#1\cdotp\cdotp\cdotp\m@th$}%
  \sbox2{$#1.\m@th$}%
  \vbox{%
    \dimen@=\wd0 %
    \advance\dimen@ -3\ht2 %
    \kern.5\dimen@
    % remove side bearings
    \dimen@=\wd2 %
    \advance\dimen@ -\ht2 %
    \dimen2=\wd0 %
    \advance\dimen2 -\dimen@
    \vbox to \dimen2{%
      \offinterlineskip
      \copy2 \vfill\copy2 \vfill\copy2 %
    }%
  }%
}
\DeclareRobustCommand\ddots{%
  \mathinner{%
    \mathpalette\@ddots{}%
    \mkern\thinmuskip
  }%
}
\newcommand*{\@ddots}[2]{%
  % #1: math style
  % #2: unused
  \sbox0{$#1\cdotp\cdotp\cdotp\m@th$}%
  \sbox2{$#1.\m@th$}%
  \vbox{%
    \dimen@=\wd0 %
    \advance\dimen@ -3\ht2 %
    \kern.5\dimen@
    % remove side bearings
    \dimen@=\wd2 %
    \advance\dimen@ -\ht2 %
    \dimen2=\wd0 %
    \advance\dimen2 -\dimen@
    \vbox to \dimen2{%
      \offinterlineskip
      \hbox{$#1\mathpunct{.}\m@th$}%
      \vfill
      \hbox{$#1\mathpunct{\kern\wd2}\mathpunct{.}\m@th$}%
      \vfill
      \hbox{$#1\mathpunct{\kern\wd2}\mathpunct{\kern\wd2}\mathpunct{.}\m@th$}%
    }%
  }%
}
\makeatother

\pgfmathdeclarefunction{gauss}{2}{%
  \pgfmathparse{1/(#2*sqrt(2*pi))*exp(-((x-#1)^2)/(2*#2^2))}%
}

\pgfmathdeclarefunction{lnormal}{2}{%
  \pgfmathparse{1/(x*#2*sqrt(2*pi))*exp(-((ln(x)-#1)^2)/(2*#2^2))}%
}

\pgfmathdeclarefunction{poisson}{1}{%
\pgfmathparse{(#1^x)*exp(-#1)/(x!)}
}

% \pgfmathdeclarefunction{gammapdf}{2}{
% \pgfmathparse{1/(#2^#1*gamma(#1))*x^(#1-1)*exp(-x/#2)}
% }

\usepgfplotslibrary{fillbetween}

\geometry{left=1.0in,right=1.0in,top=1.0in,bottom=1.0in}

\addbibresource{pep.bib}

\begin{document}

\begin{titlepage}
\title{Emulating target trials of postexposure vaccines using observational data}
\author[1]{Christopher Boyer\thanks{email: \href{mailto:cboyer@hsph.harvard.edu}{cboyer@hsph.harvard.edu}}}
\author[1,2]{Marc Lipsitch}
\affil[1]{Department of Epidemiology, Harvard T.H. Chan School of Public Health, Boston, MA.}
\affil[2]{Department of Immunology and Infectious Diseases, Harvard T.H. Chan School of Public Health, Boston, MA.}
\date{\today}
\maketitle

\begin{abstract}
Postexposure vaccination has the potential to prevent or modify the course of clinical disease among those exposed to a pathogen. However, due to logistical constraints, postexposure vaccine trials have been difficult to implement in practice. In place of trials, investigators have used observational data to estimate the effectiveness or optimal timing window for postexposure vaccines, but the relationship between these analyses and those that would be conducted in a trial is often unclear. Here, we define several possible target trials for postexposure vaccination and show how, under certain conditions, they can be emulated using observational data. We emphasize the importance of the incubation period and the timing of vaccination in trial design and emulation. As an example, we specify a protocol for postexposure vaccination against mpox and provide a step-by-step description of how to emulate it using data from a healthcare database or contact tracing program. We further illustrate some of the benefits of the target trial approach through simulation.
\noindent \\
\vspace{0in} \\
\noindent\textbf{Suggested Keywords:} target trial, vaccines, immortal time bias, effectiveness, monkeypox, postexposure prophylaxis, infectious disease \\
\end{abstract}
\setcounter{page}{0}
\thispagestyle{empty}
\end{titlepage}
\pagebreak \newpage
\doublespacing
%TC:endignore

\doparttoc % Tell to minitoc to generate a toc for the parts
\faketableofcontents % Run a fake tableofcontents command for the partocs

\part{} % Start the document part

\section{Introduction} \label{sec:introduction}
For a millennium or more humans have been inoculating healthy, unexposed individuals to prevent the onset of future disease \cite{plotkin2012vaccines}. Today, this remains the dominant paradigm for the development and mass administration of vaccines. By contrast, using vaccines to prevent clinical disease among those \textit{already exposed} to a pathogen, i.e. postexposure vaccination, remains an under-utilized strategy despite its potential to curb outbreaks and prevent the worst sequelae of disease \cite{gallagher_postexposure_2019}. This is due, in part, to the difficulty of running postexposure trials, particularly during a large outbreak, as investigators must identify, randomize, and vaccinate participants all in the, often short, time window between exposure and symptom onset. Even when these trials are feasible, the effectiveness of a postexposure vaccine is likely to vary dramatically based on the time elapsed since exposure, which can make it difficult to compare estimates across trials with different distributions of vaccination times. Finally, when there is other evidence to support effectiveness, and when other treatments are unavailable, a randomized postexposure trial may be considered unethical.

In the absence of trial data, an alternative approach is to use observational data to emulate the trial desired \cite{hernan_observational_2008,hernan_using_2016} (called a ``target trial''), for instance by using electronic healthcare records or public health contact tracing databases to define cohorts of individuals exposed to infection and comparing outcomes among those who do and do not receive post-exposure vaccination. In this paper, we define several target trials for assessing the effectiveness of postexposure vaccination depending on the causal quantity of interest (Note, in a slight abuse of terminology we refer to vaccine effectiveness rather than efficacy throughout, even when the target trial itself could reasonably be called an efficacy trial because, ultimately, the observational data used for the emulation are collected under real world conditions). We also discuss the conditions under which a trial can be emulated from observational data. We show how the target trial framework can help clarify the causal question and resolve common biases in the observational analysis of postexposure effectiveness through alignment of time zero, eligibility, and assignment as well as an unambiguous definition of the treatment strategies being contrasted. We provide an example protocol for emulating a trial of a postexposure vaccine for mpox and illustrate some of the benefits of this approach through simulation. Throughout, we focus on the direct effect of vaccination \cite{halloran_design_2010} on the individuals receiving a postexposure vaccine although the approach could be generalized to include spillover effects on others in the population.

\section{Design challenges: incubation period and timing of vaccination}
To be effective, postexposure vaccination must strike a balance between two competing forces: the incubation period of the pathogen and how long it takes to receive a vaccine. To preempt disease onset, a postexposure vaccine must stimulate an immune response faster, greater, or more specific than that provoked by natural infection alone. For example, in the case of smallpox, it is thought that a vaccine administered within 72 hours after exposure to the variola virus (the causative virus of smallpox) induces an antibody response 4 to 8 days earlier than the virus itself, most likely because the vaccine response bypasses the initial stages of natural infection in the respiratory tract \cite{massoudi_effectiveness_2003,keckler_effects_2013}. Given the incubation period for many acute infections is on the order of days, this suggests quick intervention is key. However, in real world settings, delays in receiving a vaccine are common as participants must first recognize or be notified of their exposure and then present at a healthcare clinic where a vaccine is available. 

The resulting overlap between the timing of vaccination and the timing of symptom onset creates several design challenges (see Figure \ref{fig:illustration1}). First, the effectiveness of a vaccine may vary substantially depending on how quickly participants can be vaccinated postexposure (top panel, Figure \ref{fig:illustration1}). In a randomized trial, a protocol specifies the precise vaccination strategy to be evaluated. In choosing a strategy, investigators must compromise between existing exposure identification, enrollment, and care coordination systems and what is known about the biology governing the natural course of infection. This can be difficult when the incubation period or the biological mechanism of a postexposure vaccine are not well established. Under these circumstances, a protocol may include longer delays with a secondary goal to infer the optimal postexposure window to administer the vaccine. In an observational setting, by contrast, the vaccination protocol is less clear or even absent, in which case the vaccination strategy being evaluated may be ambiguous.

Second, when vaccination is delayed there is also the possibility that some participants may have already developed symptoms prior to enrollment or vaccination. A vaccine can prevent symptoms only if administered before those symptoms start. However, when those who have symptoms at enrollment are excluded, this has implications for the population to which estimates can be generalized, as the design implicitly conditions on those who survive symptom free. When they are included, they may attenuate estimates of vaccine effectiveness relative to an ideally conducted trial as presumably vaccination post symptom onset is ineffective at preventing illness. In a trial, because eligibility is assessed prior to randomization, participants can be screened independent of their vaccination strategy and thus effect estimates remain unbiased. However, in an observational study this event is observed only among those who are vaccinated and therefore bias may result if they are differentially excluded.

Finally, a challenge specific to observational studies is the lack of an unambiguous assignment to a vaccination strategy at time zero \cite{hernan_how_2018}. In a trial, participants are explicitly assigned a strategy, e.g. vaccine or no vaccine (or placebo), at the time of enrollment and prospectively followed. By contrast in an observational study, participants are not assigned to a particular strategy. Rather their strategy is often defined retrospectively by what they do (or do not do) over the follow up period (middle panel, Figure \ref{fig:illustration1}). When there are delays in receiving a vaccine, the ambiguity creates the possibility of bias due to \textit{immortal time} among the vaccinated as they have survived symptom-free long enough to become vaccinated \cite{suissa_immortal_2008}, whereas the unvaccinated are defined independently of their survival time. 

In a trial, the challenges above can be addressed through careful design and a clear protocol. In an observational study, prospective design fixes are not generally available. However, many challenges can still be resolved through careful consideration of the trial that one would like to perform, but cannot, and then attempting to emulate it in the observational data (bottom panel, Figure \ref{fig:illustration1}).

%TC:ignore
\begin{figure}[p]
    \centering
    \begin{tikzpicture}
        \begin{axis}[
          no markers, domain=0:15, samples=100,
          axis lines*=left, xlabel=$\text{days since exposure}$, ylabel=$ $,
          title={Symptom onset times},
          height=5.5cm, width=15cm,
          xtick={1,2,3,4,5,6,7,8,9,10,11,12,13,14,15}, ytick=\empty,
          enlargelimits=false, clip=false, axis on top,
          grid = none, name=onset
          ]
          \addplot [draw=none, fill=blue!20] {lnormal(1.75,0.33)}\closedcycle;
          \addplot [very thick, blue!50!black] {lnormal(1.75,0.33)};
        %   \addplot [draw=none, fill=orange!20] {gammapdf(4, 1)}\closedcycle;
        %   \addplot [very thick, orange!50!black] {gammapdf(4, 1)};
          \addplot [very thick, red] {0.222 / (1 + exp(0.9 * (x - 4)))};
          \node[red] at (axis cs: 2.65,0.2) {$VE(t)$};
        \end{axis}
        \begin{axis}[
          at=(onset.below south west),
          anchor=north west,
          yshift=-1.3cm,
          domain=0:15, samples=100,
          axis lines*=left, xlabel=$\text{days since exposure}$, ylabel=$ $,
          title={Observational study},
          height=5.5cm, width=15cm, ymin=0, ymax=13,
          xtick={1,2,3,4,5,6,7,8,9,10,11,12,13,14,15}, ytick=\empty,
          enlargelimits=false, clip=false, axis on top,
          grid=none, y axis line style={draw=none},name=obs
          ]
          \addplot[mark=none,line width=1.2pt,dashed]
          coordinates {(0,12)(4,12)};
          \addplot[mark=none,line width=1.2pt]
          coordinates {(4,12)(15,12)};
          \addplot[mark=none,line width=1.2pt,dashed]
          coordinates {(0,11)(1,11)};
          \addplot[mark=none,line width=1.2pt]
          coordinates {(1,11)(15,11)};
          \addplot[mark=none,line width=1.2pt,dashed]
          coordinates {(0,10)(3,10)};
          \addplot[mark=none,line width=1.2pt]
          coordinates {(3,10)(15,10)};
          \addplot[mark=none,line width=1.2pt,dashed]
          coordinates {(0,9)(2,9)};
          \addplot[mark=none,line width=1.2pt]
          coordinates {(2,9)(15,9)};
          \addplot[mark=none,line width=1.2pt,dashed]
          coordinates {(0,8)(5,8)};
          \addplot[mark=none,line width=1.2pt]
          coordinates {(5,8)(15,8)};
          
          \addplot[mark=none,line width=1.2pt]
          coordinates {(0,6)(15,6)};
          \addplot[mark=none,line width=1.2pt]
          coordinates {(0,5)(15,5)};
          \addplot[mark=none,line width=1.2pt]
          coordinates {(0,4)(15,4)};
          \addplot[mark=none,line width=1.2pt]
          coordinates {(0,3)(15,3)};
          \addplot[mark=none,line width=1.2pt]
          coordinates {(0,2)(15,2)};
          \addplot[
            mark=*,
            only marks,
            mark size=3pt
            ]
            coordinates {
            (4,12)(1,11)(3,10)(2,9)(5,8)(0,6)(0,5)(0,4)(0,3)(0,2)
          };
          \addplot[
            mark=text,
            text mark=$\boldsymbol{\times}$,
            text mark as node,
            text mark style={%
                font=\large
            },
            only marks,
            mark size=5pt
            ]
            coordinates {
            (7,8)(5,10)(2,5)(9,3)(4,4)
          };
          \node[] at (axis cs: 15.35,12) {\scriptsize V};
          \node[] at (axis cs: 15.35,11) {\scriptsize V};
          \node[] at (axis cs: 15.35,10) {\scriptsize V};
          \node[] at (axis cs: 15.35,9) {\scriptsize V};
          \node[] at (axis cs: 15.35,8) {\scriptsize V};

          \node[] at (axis cs: 15.35,6) {\scriptsize C};
          \node[] at (axis cs: 15.35,5) {\scriptsize C};
          \node[] at (axis cs: 15.35,4) {\scriptsize C};
          \node[] at (axis cs: 15.35,3) {\scriptsize C};
          \node[] at (axis cs: 15.35,2) {\scriptsize C};

        \end{axis}
        \begin{axis}[
            at=(obs.below south west),
            anchor=north west,
            yshift=-1.3cm,
            domain=0:15, samples=100,
            axis lines*=left, xlabel=$\text{days since exposure}$, ylabel=$ $,
            title={Target trial emulation},
            height=5.5cm, width=15cm, ymin=0, ymax=13,
            xtick={1,2,3,4,5,6,7,8,9,10,11,12,13,14,15}, ytick=\empty,
            enlargelimits=false, clip=false, axis on top,
            grid=none, y axis line style={draw=none}
            ]
            \addplot[mark=none,line width=1.2pt,dashed]
            coordinates {(0,12)(1,12)};
            \addplot[mark=none,line width=1.2pt]
            coordinates {(1,12)(15,12)};
            \addplot[mark=none,line width=1.2pt,dashed]
            coordinates {(0,9.75)(2,9.75)};
            \addplot[mark=none,line width=1.2pt]
            coordinates {(2,9.75)(15,9.75)};
            \addplot[mark=none,line width=1.2pt,dashed]
            coordinates {(0,7.5)(3,7.5)};
            \addplot[mark=none,line width=1.2pt]
            coordinates {(3,7.5)(15,7.5)};
            \addplot[mark=none,line width=1.2pt,dashed]
            coordinates {(0,5.25)(4,5.25)};
            \addplot[mark=none,line width=1.2pt]
            coordinates {(4,5.25)(15,5.25)};
            \addplot[mark=none,line width=1.2pt,dashed]
            coordinates {(0,3)(5,3)};
            \addplot[mark=none,line width=1.2pt]
            coordinates {(5,3)(15,3)};
            
            \addplot[mark=none,line width=1.2pt,dashed]
            coordinates {(0,11)(1,11)};
            \addplot[mark=none,line width=1.2pt]
            coordinates {(1,11)(15,11)};
            \addplot[mark=none,line width=1.2pt,dashed]
            coordinates {(0,8.75)(2,8.75)};
            \addplot[mark=none,line width=1.2pt]
            coordinates {(2,8.75)(15,8.75)};
            \addplot[mark=none,line width=1.2pt,dashed]
            coordinates {(0,6.5)(3,6.5)};
            \addplot[mark=none,line width=1.2pt]
            coordinates {(3,6.5)(15,6.5)};
            \addplot[mark=none,line width=1.2pt,dashed]
            coordinates {(0,4.25)(4,4.25)};
            \addplot[mark=none,line width=1.2pt]
            coordinates {(4,4.25)(15,4.25)};
            \addplot[mark=none,line width=1.2pt,dashed]
            coordinates {(0,2)(5,2)};
            \addplot[mark=none,line width=1.2pt]
            coordinates {(5,2)(15,2)};
            \addplot[
              mark=*,
              only marks,
              mark size=3pt
              ]
              coordinates {
              (1,12)(2,9.75)(3,7.5)(4,5.25)(5,3)(1,11)(2,8.75)(3,6.5)(4,4.25)(5,2)
            };
            \addplot[
            mark=text,
            text mark=$\boldsymbol{\times}$,
            text mark as node,
            text mark style={%
                font=\large
            },
            only marks,
            mark size=5pt
            ]
            coordinates {
            (7,3)(5,7.5)(2,11)(9,4.25)(4,8.75)
          };
            \node[] at (axis cs: 15.35,12) {\scriptsize V};
            \node[] at (axis cs: 15.35,11) {\scriptsize C};
            \node[] at (axis cs: 15.35,9.75) {\scriptsize V};
            \node[] at (axis cs: 15.35,8.75) {\scriptsize C};
            \node[] at (axis cs: 15.35,7.5) {\scriptsize V};
            \node[] at (axis cs: 15.35,6.5) {\scriptsize C};
            \node[] at (axis cs: 15.35,5.25) {\scriptsize V};
            \node[] at (axis cs: 15.35,4.25) {\scriptsize C};
            \node[] at (axis cs: 15.35,3) {\scriptsize V};
            \node[] at (axis cs: 15.35,2) {\scriptsize C};
          \end{axis}
    \end{tikzpicture}
    \caption{Illustration of the challenges of evaluating postexposure vaccination using observational data. The top panel shows the distribution of symptom onset times among cases as well as vaccine effectiveness as a function of postexposure day of administration for a hypothetical pathogen. The middle panel shows an observational study with 5 vaccinated (V) and 5 unvaccinated (C) individuals in which there are delays in receiving vaccines. Dots show the time a participant's vaccination strategy at the end of follow up is first defined and Xs show symptom onset. The dashed line represents possible immortal time among vaccinated who have to survive symptom free long enough to be vaccinated. The bottom panel shows an emulation of a nested sequence of daily trials among the same individuals in which there is no immortal time bias because the timing of enrollment and ``assignment'' to a vaccination strategy coincides in each trial.}
    \label{fig:illustration1}
\end{figure}

%TC:endignore

\section{Specifying a target trial for postexposure vaccines}

\subsection{Set up and notation} \label{sec:setup}
Our goal is to estimate the effect of postexposure vaccination on the $\Delta$-day risk of clinical disease. Let the time index $t$ denote days since exposure to a case. For the purpose of emulating a target trial, we have available observational data $O=\left(L_0, A_0, D_1 \ldots, L_{\Delta-1}, A_{\Delta-1}, D_{\Delta}, X^*, T^*\right)$ on participants, where $L_t$ is a set of time-varying covariates and $L_0$ includes all covariates prior to time zero (i.e. pre-exposure). We define the following variables:
$$X: \text{ day of vaccine administration, } X^* = \operatorname{min}(X, \Delta) \text{ where } X \in \mathbb{N}+$$
$$T: \text{ day of clinical disease onset, } T^* = \operatorname{min}(T, \Delta) \text{ where } T \in \mathbb{N}+$$
$$A_t: \text{ indicator of vaccination status on day }t, A_t \in \{0, 1\}$$
$$D_t: \text{ indicator of clinical disease on day $t$, } D_t \in \{0, 1\}$$
Note that under these definitions, when $X < x$ then $A_x = 1$ and $T < \Delta$ implies $D_{\Delta} = 1$. We use days since exposure as the time scale for both vaccination time and symptom onset and administratively censor after $\Delta$ days postexposure (i.e. those unvaccinated during follow up have $X = \Delta$ and those without clinical disease have $T = \Delta$). The trial outcome $Y$ is the development of clinical disease within $\Delta$ days postexposure, i.e. $Y=D_{\Delta}$. We denote the potential outcome under an intervention which sets $A$ to $a$ by $Y^a$. For clarity, we make a few simplifying assumptions, although extensions that relax them are possible. First, we assume that the vaccine itself does not cause mild symptoms that can be mistaken for clinical disease. Second, we assume that the timing of the primary exposure event is measured without error and unambiguously defined. Third, we assume the goal of postexposure vaccination is the prevention of clinical disease in those exposed rather than reduction in disease severity or risk of further transmission, although in both cases the conceptualization of the target trial may be similar.

\subsection{Causal questions, estimands, and target trial designs}
There are multiple causal questions about the effects of postexposure vaccines that are of interest to patients, clinicians, and policymakers. Each question potentially requires a different causal quantity (i.e. a causal contrast between vaccine strategies - or estimand), and therefore a different hypothetical trial, to answer it. In practice, not all trials will be equally feasible or ethical to implement in which case the ideal  casual quantity may be replaced by a more feasible one. An advantage of emulating a trial from observational data is that we can potentially target any quantity. However, unbiased estimation will crucially depend on whether the strong untestable assumptions underlying the emulation hold. Below we discuss three potential causal questions about postexposure vaccines and the corresponding estimand and target trial that would answer them with additional details provided in \ref{sec:additional_design}. 

Under the theory that the earlier a vaccine is administered postexposure the better, the ideal vaccination strategy is likely to vaccinate immediately upon exposure to the pathogen. A natural question, therefore, is \textit{what is this maximum effectiveness}? In this case, a possible estimand is the contrast
\begin{equation}
  VE(0) = 1 - \frac{\Pr[Y^{x = 0} = 1]}{\Pr[Y^{x > \Delta} = 1]}
\end{equation}
where the numerator is the risk of symptoms within $\Delta$ days under immediate vaccination and the denominator is the risk under no vaccination over follow up (Note, using our definition of time-varying treatment $A_t$, we could alternatively write this as $VE(0) = 1 - \Pr[Y^{\overline{a}_{K} = 1} = 1]/\Pr[Y^{\overline{a}_{K} = 0} = 1]$ where $K = \Delta - 1$). Assuming perfect adherence, $VE(0)$ could be estimated in a trial in which eligible participants are recruited immediately upon exposure, randomized to vaccine or no vaccine, and then the $\Delta$-day cumulative incidence of symptoms in the two groups are compared (we discuss estimating vaccine effectiveness based on the hazard ratio rather than cumulative incidence in \ref{sec:effect_measures}). 

Alternatively, one could ask: \textit{how effective would a vaccine be if administered after a $t$-day delay}? In this case, the estimand of interest is the $t$-specific contrast
\begin{equation}
VE(t) = 1 - \frac{\Pr[Y^{x = t} = 1]}{\Pr[Y^{x > \Delta} = 1]}.
\end{equation}
To estimate $VE(t)$, one might consider a design in which participants are still enrolled immediately postexposure and randomized to vaccine or no vaccine, but then further randomly assigned a day that they are to receive a vaccine. Such a design permits the estimation of the maximum delay window beyond which population effectiveness falls below a minimum threshold (see \ref{sec:maxdelay}), which may be useful for public health planning. We refer to this design and the previous one as ``day zero'' designs as they contrast strategies that are assigned at postexposure day zero. Neither is common in practice as they require enrolling participants immediately after exposure which is infeasible.

From the perspective of the exposed individual or their clinician, a more useful question might be: \textit{given that an individual presents symptom free on day $t$ what is the effectiveness of the vaccine if administered immediately}? Here a relevant casual estimand is the $t$-specific vaccine effectiveness among those unvaccinated and surviving symptom-free to day $t$, i.e.
\begin{equation}
  VE_{cond}(t) = 1 - \frac{\Pr[Y^{x = t} = 1 \mid X \geq t, T > t]}{\Pr[Y^{x > \Delta} = 1 \mid X \geq t, T > t]}
\end{equation}
To estimate $VE_{cond}(t)$, one could specify a fixed time window that participants are eligible for enrollment and randomize them on the postexposure day they present (hereafter ``fixed-enrollment period'' design). Given that length of delay is likely a strong determinant of effectiveness, we could improve efficiency by matching or blocking eligible participants on the postexposure day they present and randomizing within enrollment strata. An advantage of this design is that it allows individuals to present ``naturally'' rather than on a specified day. Note that, in general, the $t$-specific vaccine efficacies, $VE_{cond}(t)$, will not be the same as the $VE(t)$ defined previously as they are conditional on surviving symptom-free. Because participants are allowed to present naturally, those that present earlier may be systematically different than those presenting later with respect to their risk of developing clinical disease. Indeed, $VE(t)$ and $VE_{cond}(t)$ will only coincide when there is no effect modification by enrollment day or symptom onset time, which are both implausible.

A final design possibility is to enroll participants immediately but allow a \textit{grace period} \cite{smith_emulation_2022,wanis_role_2022}, i.e. a fixed time window after randomization in which vaccination can be initiated. We discuss designs that allow for a grace period further in \ref{sec:additional_design}. 

\subsection{Example protocol for a target trial of a postexposure Mpox vaccine}

To illustrate the specification of a postexposure target trial, here we outline a protocol for a trial to evaluate the effectiveness of the JYNNEOS vaccine as postexposure prophylaxis against development of symptomatic mpox infection. For ease of exposition, we focus on a single design: the fixed-enrollment period design.

Briefly, for context the human mpox virus (MPXV) is an orthopox virus and related to the virus that causes smallpox. In April 2022, an outbreak of mpox occurred in several countries prompting the World Health Organization to declare a public health emergency of international concern \cite{nuzzo_who_2022}. A two-dose live replicating vaccine for smallpox and mpox (MVA-BN), licensed as JYNNEOS\textsuperscript{TM}, was approved by the Food and Drug Administration (FDA) in 2019. During the outbreak, the vaccine was offered as postexposure prophylaxis to contacts of confirmed mpox cases. In guidance documents, the U.S. Centers for Disease Control and Prevention (CDC) recommended that unvaccinated people exposed to the mpox virus be vaccinated with a first vaccine dose against mpox within 4 days of exposure for the greatest likelihood of preventing disease \cite{cdc_mpox_2023-1}, though also suggested there may still be benefit to vaccination within 14 days of exposure \cite{kecmanovic1975einfluss,sommer_1972_1974}. Licensure of JYNNEOS was supported by animal studies \cite{earl_rapid_2008,keckler_effects_2013,hatch_assessment_2013,samuelsson_survival_2008} and immunogenicity studies \cite{pittman_phase_2019} but to date no trial data on the postexposure effectiveness of the vaccine against mpox exists. Therefore, an emulation of a postexposure trial using observational data may provide useful evidence for setting policy.

Table \ref{tab:protocol} summarizes the components of a target trial of the JYNNEOS vaccine. A brief description of each component is provided in \ref{sec:additional_design}.

%TC:ignore
\begin{table}[p]
    \small
    \centering
    \caption{Example protocol for the specification and emulation of a target trial of postexposure vaccination for prevention of mpox.\label{tab:protocol}}
    \begin{threeparttable}
    \begin{tabular}{>{\raggedright\arraybackslash}p{2.5cm}>{\raggedright\arraybackslash}p{7.75cm}>{\raggedright\arraybackslash}p{5cm}}
        \toprule
        Protocol component & Target trial specification & Emulation \\
        \midrule
        Eligibility & \makecell*[t{{>{\raggedright\arraybackslash}p{7.5cm}}}]{
            High\textsuperscript{a} or intermediate\textsuperscript{b} risk exposure to a PCR-confirmed mpox case within the first 14 days postexposure AND negative PCR for mpox or orthopox virus at enrollment AND no symptoms AND no prior history of JYNNEOS vaccination } & same \\
            & & \\
        Treatment strategies & \makecell*[t{{>{\raggedright\arraybackslash}p{7.5cm}}}]{
            (1) JYNNEOS vaccination immediately upon enrollment \\
            (2) no JYNNEOS vaccination during 21 days postexposure} & same  \\
        & & \\
        Treatment assignment & non-blinded random assignment to either strategy (1) or (2) at enrollment & same but randomization is emulated by conditioning on covariates and time since exposure  \\
        & & \\
        Outcomes & 21-day cumulative incidence of  disease defined as symptom onset and PCR-confirmed mpox or orthopox & same \\
        & & \\
        Time zero & Day enrolled and assigned a strategy during the 14-day fixed enrollment period starting from exposure date & emulated as a sequence of nested trials starting each of the 14-day enrollment period \\
        & & \\
        Follow up & Start at time zero and follow until clinical disease onset, loss to follow up, or 21 days have elapsed, whichever is first & same  \\
        & & \\
        Causal contrast & \makecell*[t{{>{\raggedright\arraybackslash}p{7.5cm}}}]{Intention-to-treat $VE_{cond}(t)$ \\ Per protocol $VE_{cond}(t)$} & observational analog of per protocol $VE_{cond}(t)$  \\
        & & \\
        Statistical analysis & \makecell*[t{{>{\raggedright\arraybackslash}p{7.5cm}}}]{Intention-to-treat: compare cumulative incidence of clinical disease under each strategy, adjusting for loss to follow up and prognostic factors to increase efficiency \\ 
        \\ Per protocol: Use IPW/g-formula/ g-estimation to account for non-adherence.} &  same as per protocol  \\
        \bottomrule
    \end{tabular}
    \begin{tablenotes}
        \item[a] \textit{High risk:} direct mucosal or broken skin contact with lesions or bodily fluids OR any sexual or intimate mucosal contact OR indirect mucosal or broken skin contact with lesions or bodily fluids via linens, clothing, or other materials.
        \item[b] \textit{Intermediate risk:} unmasked exposure to respiratory droplets (within 6 ft for $>$3 hours) OR direct contact between intact skin and lesions or bodily fluids OR indirect contact between intact skin and lesions or bodily fluids via linens, clothing, or other materials OR indirect contact between exposed individual's clothing with linens or bodily fluids.
    \end{tablenotes}
\end{threeparttable}
\end{table}
%TC:endignore

\section{Emulation of a postexposure vaccine trial}

Once the target trial is specified, we can attempt to emulate it using observational data. Emulating a postexposure vaccination trial will generally require linking high quality case and contact surveillance with clinical databases or registries recording vaccinations as well as intensive post vaccination symptom monitoring. Here, we outline how to emulate the main components of the target trial as well as common challenges, using the JYNNEOS vaccine example to help ground our discussion. Additional details on the identifiability conditions (\ref{sec:identifiability}), data setup/manipulation (\ref{sec:datamanip}), and estimation (\ref{sec:estimation}) steps necessary to emulate all trial designs discussed above are available in the Web Appendix.

\subsubsection*{Eligibility}
Ideally, eligibility criteria in the emulation should match those in the target trial. In particular, this means we cannot include restrictions based on post-baseline events (e.g. ``exclude those vaccinated more than 15 days after exposure or those vaccinated after symptoms'') as these may introduce bias and would be unavailable at baseline in the target trial. Additional challenges may arise because there is no direct contact with participants at enrollment. Rather we must rely on routinely collected data which may not be fit-for-purpose. For instance, we may have to assume that those without a previous vaccination in the electronic medical records database did not receive a vaccine from a different healthcare system.

More broadly, when emulating postexposure trials, determining eligibility requires knowing who is actually at risk of infection. This means proper classification of those exposed to an index case is needed as well as an accurate history of vaccination or previous infection and screening for symptoms or PCR-positivity at enrollment. Infection history may be spotty if it mostly consists of prior recorded infections unless the pathogen is novel or invades a mostly naive population. Vaccination history may come from medical records or vaccination registries. Ideally, contacts of the index case would all be offered PCR testing upon notification of exposure and then enrolled in active symptom tracking, such as through daily phone calls or text messages, as this would prevent differential eligibility assessments of vaccinated and unvaccinated participants. However, in practice, investigators may have to assume that the lack of a positive PCR test and/or no passive symptom report constitutes no infection at time eligibility is assessed in the emulation.

\subsubsection*{Treatment strategies}
The vaccination strategies to be emulated should also match those in the target trial. As participants in observational data sets will almost always be aware of their treatment strategy, the trial emulated will typically be a pragmatic (unblinded) trial. To emulate a target trial, we identify individuals in the database who meet all of the eligibility criteria. We then assign them to the strategy or strategies that are consistent with their observed data at baseline.

To properly ``assign'' participants to strategies in the emulation, accurate data on the postexposure timing of vaccination is necessary. This will also allow us to censor them when they deviate from their assigned protocol. In order to identify the unvaccinated, we must inevitably assume that those without vaccinations recorded in a registry or health records truly did not receive a vaccine during follow up. This may be a problem if participants can receive care from sources not covered by study data. 

Another challenge is that, to properly define regimes, the exposure date should be accurate and unambiguously defined. The accuracy of exposure information may depend on the salience of the event and the ability of index cases or their contacts to recall interactions. An unambiguous definition requires a detailed description of what constitutes possibly infectious contact informed by the underlying biology. In our mpox example, this description comes from guidance published by the CDC, but may not be as clear for other pathogens. Participants may also be exposed multiple times or over an extended duration, in which case determining which time to set as the definitive exposure date may be less clear. As a sensitivity analysis we might consider multiple alternative definitions.

\subsubsection*{Assignment procedures}
In the emulation, allocation to treatment strategies is assumed to be random conditional on a sufficient set of covariates to control confounding. For postexposure vaccination against mpox this may include time since exposure, risk level of contact with index case, calendar week, geographic region, age, sex, gender, coexisting conditions affecting immune system (e.g. HIV or STIs, obesity, cancer, immune suppressing therapies), and proxies for healthcare utilization (e.g. flu vaccination, outpatient visits, HIV-PrEP).

In practice, our ability to correctly estimate effects will depend on the conditional randomization assumption, at least approximately, holding (equivalent to assuming that there is little residual confounding). If those who access postexposure vaccines are those with higher risk exposures to mpox or with weaker immune systems (along some dimension not captured by the covariates) then we will likely underestimate the true effectiveness of the vaccine. On the other hand, if those who access postexposure vaccines are healthier and more likely to engage in healthy behaviors more broadly (again along dimensions not captured by the covariates), then we will likely overestimate the true effectiveness of the vaccine. The availability of rich covariate information on participants as well as deep subject matter knowledge about the determinants of both who gets vaccinated and the clinical course of disease are essential.

While direct verification of this assumption is not possible, several design and analytic strategies can limit or quantify the bias that would result from violations. One strategy is to identify possible negative outcome controls \cite{chua_use_2020,lipsitch_negative_2010-1}, that is outcomes where confounding structure is expected to be similar but are plausibly unaffected by vaccination. For instance, routine visits for other conditions may be a proxy for unmeasured health-seeking behaviors or testing positive for syphilis may be a proxy for unmeasured high-risk sexual behavior.  Another strategy is to conduct a sensitivity analysis to quantify the potential bias by evaluating change in estimated effect across a plausible range of parameter values dictating the strength of unmeasured confounding \cite{robins_sensitivity_2000}. 

\subsubsection*{Outcome}
Outcome definitions and measurements should be as similar to those in the target trial as possible. In a postexposure vaccine trial, there is often a regular system for monitoring of symptoms over the follow up period. In an observational emulation, this data may be passively collected, leaving the opportunity for potential outcome missclassification, particularly when there is a mild form of the disease which may go unnoticed or unreported or when participants may seek care from providers not covered by study data sources. This may be less of a concern when cases are reportable or the pathogen is novel. Existing symptom monitoring systems may be in place as part of contact tracing and testing systems in which case they can be leveraged. Ideally, ascertainment of symptoms would be blind to an individual's vaccination status. If those who are vaccinated are better surveilled or use passive systems more frequently this could introduce bias. 

\subsubsection*{Causal contrast}
In theory the contrasts will be the same as in the target trial, although in some instances a corollary of the intention-to-treat effect may not be estimable from the observational data. Here, we focus on the per-protocol \cite{hernan_per-protocol_2017} analysis of $VE_{cond}(t)$.

\subsubsection*{Statistical analysis}
Compared to the analyses in the target trial, the analyses in the emulation are complicated by two factors. First, randomization is assumed to only hold conditional on covariates. Therefore our analysis must include an appropriate method of adjustment such as outcome regression, matching, inverse-probability weighting, or a combination thereof. 

Second, unlike in a trial, in an emulation the assigned strategy at baseline is not known, rather it must be inferred from the observed data. In particular, participants are not assigned to vaccine or no vaccine at time zero. To avoid immortal time bias, we need to choose a start of follow up in the emulation in a way that ensures that the distribution of time since exposure is the same in both groups \cite{hernan2016specifying}. In the fixed enrollment period design, this can be accomplished via emulating nested daily sequential trials: starting from exposure date, each day we identify participants who are eligible to participate (e.g. no prior vaccination or mpox infection) and assign those receiving a vaccine on that day to the vaccine strategy and those who do not receive a vaccine on that day to the no vaccine strategy (see \ref{sec:datamanip}). In this setup, unvaccinated participants will be eligible to serve as controls in multiple trials until they receive a vaccine or develop symptoms. To estimate per protocol effects, we censor participants when their data deviates from their ``assigned'' regime and then adjust for possible time-varying selection bias using any g-method such as inverse-probability of censoring weights (see \ref{sec:estimation}). Additionally, because we are using the same participant in multiple nested trials our observations are no longer independent. Therefore, appropriate adjustment to our standard errors is necessary to account for possible correlation across observations. Adjustment can be made either by using a cluster-robust variance estimator or the bootstrap.  

\section{Simulation}\label{sec:simulation}
%TC:ignore
\begin{figure}[t]
  \centering
  \includegraphics{../../../../3_figures/dist.pdf}
  \caption{Distribution of simulated vaccination times ($X^*$) among vaccinated and symptom onset times ($T^*$) among cases when $VE = 0$ over the 21 days of follow up showing the degree of overlap.}
  \label{fig:example_overlap}
\end{figure}
%TC:endignore

To demonstrate the benefits of the target trial approach, we simulated data from hypothetical observational study under a known data generation process in which there is an overlap between vaccination timing and the timing of symptom onset (full details in \ref{sec:simulation_appendix}). Figure \ref{fig:example_overlap} shows an example of the overlap when $VE = 0$. We used this setup to compare explicit emulation of a target trial with a few common estimation strategies drawn from the literature. Specifically, we compare the following strategies:
    \begin{enumerate}
        \item \textit{naive, leave} - a simple comparison of the ``ever vaccinated'' and ``never vaccinated'' using the relative risk regression model $\Pr[Y = 1 \mid X] = \operatorname{exp}\{\beta_0 + \beta_1 I(X < 21)\}$ with vaccine effectiveness estimated as $\widehat{VE} = 1 - \exp(\widehat{\beta_1})$.
        \item \textit{naive, move} - re-classify those who receive vaccine after developing symptoms  as ``unvaccinated'', i.e. we use the relative risk regression model $\Pr[Y = 1 \mid X] = \operatorname{exp}\{\beta_0 + \beta_1 I(X < T)\}$ where $I(X<T)$ implies only those who receive vaccine prior to symptom onset are ``vaccinated'' with vaccine effectiveness again estimated as $\widehat{VE} = 1 - \exp(\widehat{\beta_1})$.
        \item \textit{target trial} - we emulate a sequence of nested daily trials as described above and in  \ref{sec:datamanip}. In each trial, we censor participants when they deviate from their assigned strategy at baseline and use inverse-probability of censoring weights to adjust for selection bias. These nested trials are combined, and vaccine effectiveness is estimated using standardized cumulative incidence curves from a pooled logistic regression.
    \end{enumerate}
    
In Figure \ref{fig:sim_results} we compare estimates of $VE$ to the truth across two scenarios: the first when the true $VE = 0\%$ and the second when the true $VE = 31.6\%$. Under the null, the naive approaches are upwardly biased due to immortal time bias (i.e. by definition vaccinated have to survive long enough to be vaccinated while unvaccinated are at risk at all time points), while the target trial approaches yield valid estimates. This persists in scenario 2 where $VE = 31.6\%$, although the relative bias of the first approach is somewhat offset by the fact that those vaccinated after developing symptoms are included with vaccinated. In \ref{sec:simulation_appendix}, we also compare estimation strategies when $VE$ varies with timing of vaccination (Figure \ref{fig:sim_hetx}), when effectiveness is defined as one minus the hazard ratio and/or a time-varying cox model is used (Figure \ref{fig:sim_hr}), and when the degree of overlap between vaccination and symptom onset is varied (Figure \ref{fig:sim_overlap}). Of note, in this simple scenario both the target trial emulation approach and the time-varying cox model yielded unbiased estimates of VE when defined as one minus the hazard ratio.

%TC:ignore

\begin{figure}[t]
  \centering
  \includegraphics{../../../../3_figures/sim_rr.pdf}
  \caption{Simulated $VE$ estimates compared to the truth for the three estimation strategies described in section \ref{sec:simulation}. Naive, leave refers to simple comparison of those ever vaccinated over follow up versus never vaccinated. Naive, move is also a comparison of ever versus never vaccinated but those who receive vaccine after symptoms start are re-classified as never vaccinated. Target trial refers to the approach of emulating a sequence of nested trials as discussed in the main text. Based on 1000 monte carlo simulations. Dashed line shows true value in each scenario. \label{fig:sim_results}}
  \end{figure}
%TC:endignore

\section{Discussion} \label{sec:discussion}
Accurate assessments of postexposure effectiveness of vaccines could be useful for curbing the worst sequelae of many pathogens, but trials are often infeasible or unethical. Here, we specified target trials for postexposure vaccination and describe how to emulate them using observational data. Using the example of mpox vaccines, we discussed some of the unique challenges of emulating postexposure vaccination trials, including the central role played by the distribution of vaccination times and the incubation period. Throughout we emphasize the clarifying role of the target trial framework and conclude with simulations showing how emulating the trial can help avoid several common biases in observational analyses. 

Previous studies have emulated trials of pre-exposure vaccines, particularly during the COVID-19 pandemic \cite{dagan_bnt162b2_2021,dickerman_comparative_2022,cohen-stavi_bnt162b2_2022,barda_effectiveness_2021}. These studies filled gaps in the literature by emulating trials which were not feasible to implement in practice such as head-to-head comparisons of vaccines \cite{dickerman_comparative_2022}, effectiveness against new variants \cite{cohen-stavi_bnt162b2_2022}, effectiveness of boosters \cite{barda_effectiveness_2021,magen_fourth_2022}, and effectiveness in important subgroups such as children \cite{cohen-stavi_bnt162b2_2022} and the immunocompromised. Observational emulations of post-exposure vaccines could perform a similar function.
 
We considered postexposure trials where the goal of vaccination is to prevent the onset of clinical disease. However, other goals such as reducing severity or transmission are also possible. In rich observational datasets multiple primary and secondary endpoints may be feasible. To emulate trials in which the goal is to reduce severity, one could simply replace onset with an alternative outcome such as hospitalization or death due to disease of interest in the trials outlined above. Emulating a trial of the effect of postexposure vaccination on transmission would require close follow up or even random testing of the contacts of the vaccinated and unvaccinated participants and may be compromised by changes in exposure behaviors due to lack of blinding in most observational settings. However, if PCR tests were administered to everyone independent of symptoms, effectiveness against infection (PCR-positivity) is a lower bound on effectiveness against transmission \cite{lipsitch_interpreting_2021}. 

%Beyond estimating postexposure effectiveness, another goal of a postexposure trial could be to determine the maximum vaccination delay before effectiveness falls below a certain cost-benefit threshold. This quantity is important both for policymakers communicating with high-risk groups and the broader public about what to do in the event of an exposure as well as to help practitioners determine whether vaccination is still indicated. Given sufficient sample size, estimates of ``real world'' effectiveness of postexposure vaccination could be produced for several possible vaccination delay strategies. In section \ref{sec:maxdelay} of the Appendix, we develop a formal counterfactual framework for the maximum delay and provide additional details on how to estimate it using data from an observational emulation.

\clearpage

\newpage
%TC:ignore

\printbibliography

\clearpage

\documentclass[11pt]{article}

\usepackage{
    amssymb,
    amsmath,
    amsfonts,
    calc,
    eurosym,
    geometry,
    ulem,
    graphicx,
    caption,
    color,
    setspace,
    sectsty,
    comment,
    footmisc,
    caption,
    % natbib,
    pdflscape,
    subcaption,
    subfiles,
    titling,
    array,
    hyperref,
    booktabs,
    longtable,
    float,
    authblk,
    makecell,
    threeparttable,
    pgfplots,
    minitoc,
    tocbasic}

\usepackage[page]{appendix} % print appendices title

\usepackage[
    backend=biber,
    style=nature,
    date=year,
    doi=true,
    isbn=false,
    url=false,
    eprint=false
]{biblatex}

% Make the "Part I" text invisible
\renewcommand \thepart{}
\renewcommand \partname{}
\DeclareTOCStyleEntry[dynnumwidth=true,
                      numsep=1em,
]{tocline}{section}
\DeclareTOCStyleEntry[dynnumwidth=true,
                      numsep=1em,
]{tocline}{subsection}
\AtEveryBibitem{%
  \clearfield{note}%
}
\AtEveryCitekey{\clearlist{publisher}}
\AtEveryBibitem{\clearlist{publisher}}

\usepackage{pgf,tikz}
\usetikzlibrary{arrows, automata}
\usetikzlibrary{shapes.geometric}
\usetikzlibrary{positioning,calc, decorations.pathreplacing}

\usepackage{siunitx}
\newcolumntype{d}{S[input-symbols = ()]}

\normalem

\renewcommand\Affilfont{\small\itshape}

\onehalfspacing
\newtheorem{theorem}{Theorem}
\newtheorem{corollary}[theorem]{Corollary}
\newtheorem{proposition}{Proposition}
\newenvironment{proof}[1][Proof]{\noindent\textbf{#1.} }{\ \rule{0.5em}{0.5em}}

\newtheorem{hyp}{Hypothesis}
\newtheorem{subhyp}{Hypothesis}[hyp]
\renewcommand{\thesubhyp}{\thehyp\alph{subhyp}}

\newcommand{\red}[1]{{\color{red} #1}}
\newcommand{\blue}[1]{{\color{blue} #1}}

\newcolumntype{L}[1]{>{\raggedright\arraybackslash}m{#1}}
\newcolumntype{C}[1]{>{\centering\arraybackslash}m{#1}}
\newcolumntype{R}[1]{>{\raggedleft\arraybackslash}m{#1}}
\subsubsectionfont{\normalfont\itshape}

\usepackage{mathtools}

\usepackage{letltxmacro}
\LetLtxMacro\orgvdots\vdots
\LetLtxMacro\orgddots\ddots

\makeatletter
\DeclareRobustCommand\vdots{%
  \mathpalette\@vdots{}%
}
\newcommand*{\@vdots}[2]{%
  % #1: math style
  % #2: unused
  \sbox0{$#1\cdotp\cdotp\cdotp\m@th$}%
  \sbox2{$#1.\m@th$}%
  \vbox{%
    \dimen@=\wd0 %
    \advance\dimen@ -3\ht2 %
    \kern.5\dimen@
    % remove side bearings
    \dimen@=\wd2 %
    \advance\dimen@ -\ht2 %
    \dimen2=\wd0 %
    \advance\dimen2 -\dimen@
    \vbox to \dimen2{%
      \offinterlineskip
      \copy2 \vfill\copy2 \vfill\copy2 %
    }%
  }%
}
\DeclareRobustCommand\ddots{%
  \mathinner{%
    \mathpalette\@ddots{}%
    \mkern\thinmuskip
  }%
}
\newcommand*{\@ddots}[2]{%
  % #1: math style
  % #2: unused
  \sbox0{$#1\cdotp\cdotp\cdotp\m@th$}%
  \sbox2{$#1.\m@th$}%
  \vbox{%
    \dimen@=\wd0 %
    \advance\dimen@ -3\ht2 %
    \kern.5\dimen@
    % remove side bearings
    \dimen@=\wd2 %
    \advance\dimen@ -\ht2 %
    \dimen2=\wd0 %
    \advance\dimen2 -\dimen@
    \vbox to \dimen2{%
      \offinterlineskip
      \hbox{$#1\mathpunct{.}\m@th$}%
      \vfill
      \hbox{$#1\mathpunct{\kern\wd2}\mathpunct{.}\m@th$}%
      \vfill
      \hbox{$#1\mathpunct{\kern\wd2}\mathpunct{\kern\wd2}\mathpunct{.}\m@th$}%
    }%
  }%
}
\makeatother

\pgfmathdeclarefunction{gauss}{2}{%
  \pgfmathparse{1/(#2*sqrt(2*pi))*exp(-((x-#1)^2)/(2*#2^2))}%
}

\pgfmathdeclarefunction{lnormal}{2}{%
  \pgfmathparse{1/(x*#2*sqrt(2*pi))*exp(-((ln(x)-#1)^2)/(2*#2^2))}%
}

\pgfmathdeclarefunction{poisson}{1}{%
\pgfmathparse{(#1^x)*exp(-#1)/(x!)}
}

% \pgfmathdeclarefunction{gammapdf}{2}{
% \pgfmathparse{1/(#2^#1*gamma(#1))*x^(#1-1)*exp(-x/#2)}
% }

\usepgfplotslibrary{fillbetween}

\geometry{left=1.0in,right=1.0in,top=1.0in,bottom=1.0in}

%\addbibresource{lipids.bib}

\begin{document}

\begin{appendix}

    \renewcommand{\thefigure}{A\arabic{figure}}
    \setcounter{figure}{0}
    
    \renewcommand{\thetable}{A\arabic{table}}
    \setcounter{table}{0}
    
    \renewcommand{\theequation}{A\arabic{equation}}
    \setcounter{equation}{0}

%    \appendixwithtoc
    \newpage

    \section{Appendix}

    \subsection{Grace period design}

Another possibility is to specify a \textit{grace period}, i.e. a fixed time window in which vaccination can be initiated. For example, in a postexposure trial of a varicella vaccine, the investigators stipulated that sibling contacts of varicella case were ``were identified by their primary pediatrician and referred to our department \textit{within 72 hours} of the appearance of the first skin lesion'' in the index case. Under this design, the causal target would be the average vaccine effectiveness during the $\delta$ days of the grace period, i.e.
$$
\overline{VE}_\delta = 1 - \frac{\Pr[Y^{g(X,\delta)} = 1]}{\Pr[Y^{x > \Delta} = 1]}
$$ 
where
\begin{gather*}
    g(X,\delta): \text{get vaccinated within $\delta$ days of exposure under the expected vaccine } \\ \text{ administration pattern }  f^*(X \mid \overline{L}_t, X > t, T > t)
\end{gather*}
where, for instance, in the varicella trial $\delta = 3$. Importantly, implicit in this definition is that, when effectiveness varies by the time since exposure, as it most certainly does for most postexposure vaccines, such a design estimates the average effectiveness \textit{under the natural time course of vaccination}, $f^*(X \mid \overline{L}_t, X > t, T > t) = f(X \mid \overline{L}_t, X > t, T > t)$. This implies that two trials identical in all respects except for the distribution of vaccinations over the grace period could yield substantially different estimates. Therefore, a trialist pursuing this design has to strike a balance when defining a grace period between ensuring the period is short enough that benefit is immunologically possible and the trial is adequately powered, but also long enough that the regime is clinically feasible under reasonable assumptions about how quickly patients are notified of their exposure to a case and can access a vaccine in the real world. Properly conceived a grace period design can provide evidence about average effectiveness of postexposure vaccination administered within a certain window under real world conditions. As such it may be a more useful estimate for population planning or modeling studies than those produced by the stratified design above. When there's no effect modification by initiation or onset time, the average effectiveness is equal to $VE(t)$ standardized over the distribution of vaccine administration times during the grace period, i.e.
$$\overline{VE}_\delta = \sum_{t = 1}^\delta VE(t) \times f^*_X(t \mid \overline{L}_t, X > t, T > t).$$

    \subsection{Determining maximum postexposure vaccination delay}
    In setting guidelines for postexposure vaccination, a common problem for practitioners and policymakers is to determine the maximum delay before efficacy falls below a certain threshold. Absent clear biology or immune response data, this can be difficult to determine empirically even when postexposure trials are possible as trial participants are generally only randomly assigned to vaccine or no vaccine/placebo not to a specific day. In this section, we suggest a method for estimating the maximum delay based on a pre-specified minimum efficacy bound. In principle, this method could be applied either in a randomized trial where the day of vaccination is not strictly controlled or in an observational emulation. 
 
    Two thresholds:
    \begin{enumerate}
        \item At exposure day zero, what's the maximum delay time before efficacy drops below a certain threshold.
        \item Conditional on surviving symptom free to that point, what's the last day that the efficacy of getting vaccinated versus not is above a minimum threshold.
    \end{enumerate}
    The second question is only relevant to people who have survived symptom free up, the first is probably better population-level guidance on how quickly people should try to get vaccinated 

    To fix concepts, let 

    We propose to estimate the parameters $\psi$ of the marginal structural model 
    $$\Pr[D^{x}_{t+1} = 1 \mid \overline{D}^x_t = 0] =  \operatorname{expit}\{\psi_{0,t} + \psi_1 f(x) + \psi_2 f(x) t\}$$
    where $f(x)$ is a flexible function of the day of vaccination such as a restricted cubic spline and $f(x)t$ is the product of $f(x)$ and time since postexposure
    
    $$VE^x(t) = 1 - \frac{\sum_{t=0}^\Delta \Pr[D^{x}_{t+1} = 1 \mid \overline{D}^x_t = 0] \prod_{j=0}^t\Pr[D^{x}_{j+1} = 0 \mid \overline{D}^x_j = 0]}{\sum_{t=0}^\Delta \Pr[D^{\infty}_{t+1} = 1 \mid \overline{D}^\infty_t = 0] \prod_{j=0}^t\Pr[D^{\infty}_{j+1} = 0 \mid \overline{D}^\infty_j = 0]}$$

    $$\Pr[D^{a}_{t+1} = 1 \mid \overline{D}^a_t = 0, X = x] =  \operatorname{expit}\{\psi_{0}f(x + t) + \psi_1 a + \psi_2 a f(x + t) \}$$

    \subsection{Example emulation from hypothetical dataset}


    \begin{figure}[p]
    \centering
    \begin{tikzpicture}[> = stealth, shorten > = 1pt, auto, node distance = 1.5cm, thick]
    
    \node[circle,draw=none] (y0) {$Y$};
    \node[circle,draw=none] (y1) [below of=y0] {$Y$};
    \node[circle,draw=none] (y2) [below of=y1] {$Y$};
    \node[circle,draw=none] (y3) [below of=y2] {$Y$};
    \node[circle,draw=none] (d) [below of=y3] {\large $\vdots$};
    \node[circle,draw=none] (y4) [below of=d] {$Y$};
    \node[circle,draw=none] (y5) [below of=y4] {$Y$};

    \node[circle,draw,minimum size =1cm] (a03) [left of=y0] {$1$};
    \node[circle,draw,minimum size =1cm] (a02) [left of=a03] {$1$};
    \node[circle,draw,minimum size =1cm] (a01) [left of=a02] {$1$};
    \node[circle,draw,minimum size =1cm] (a0)  [left of=a01] {$1$};

    \node[circle,draw,minimum size =1cm] (a13) [left of=y1] {$0$};
    \node[circle,draw,minimum size =1cm] (a12) [left of=a13] {$0$};
    \node[circle,draw,minimum size =1cm] (a11) [left of=a12] {$0$};
    \node[circle,draw,minimum size =1cm] (a1)  [left of=a11] {$0$};

    \node[regular polygon,regular polygon sides=4, draw,fill=gray!30,minimum size =1.25cm] (l0) [below left=0.01cm and 1.5cm of a0] {$ $};  
    
    \node[circle,draw=none] (day0) [left= 1cm of l0] {$t = 0$};

    \path[->] (l0) edge node {} (a0);
    \path[->] (l0) edge node {} (a1);
    \path[->] (a0) edge node {} (a01);
    \path[->] (a1) edge node {} (a11);

    \path[->] (a11) edge node {} (a12);
    \path[->] (a12) edge node {} (a13);
    \path[->] (a13) edge node {} (y1);

    \path[->] (a01) edge node {} (a02);
    \path[->] (a02) edge node {} (a03);
    \path[->] (a03) edge node {} (y0);

    \node[circle,draw,minimum size =1cm] (a23) [left of=y2] {$1$};
    \node[circle,draw,minimum size =1cm] (a22) [left of=a23] {$1$};
    \node[circle,draw,minimum size =1cm] (a21) [left of=a22] {$1$};

    \node[circle,draw,minimum size =1cm] (a33) [left of=y3] {$0$};
    \node[circle,draw,minimum size =1cm] (a32) [left of=a33] {$0$};
    \node[circle,draw,minimum size =1cm] (a31) [left of=a32] {$0$};

    \node[regular polygon,regular polygon sides=4, draw,fill=gray!30,minimum size =1.25cm] (l11) [below left=0.01cm and 1.5cm of a21] {$ $};  
    \node[regular polygon,regular polygon sides=4, draw,fill=gray!30,minimum size =1.25cm] (l10) [left of=l11] {$ $};  

    \node[circle,draw=none] (day1) [left= 1cm of l10] {$t = 1$};

    \path[->] (l11) edge node {} (a21);
    \path[->] (l11) edge node {} (a31);

    \path[->] (a21) edge node {} (a22);
    \path[->] (a22) edge node {} (a23);
    \path[->] (a23) edge node {} (y2);

    \path[->] (a31) edge node {} (a32);
    \path[->] (a32) edge node {} (a33);
    \path[->] (a33) edge node {} (y3);

    \node[circle,draw=none] (dd) [below of=a32] {\large $\ddots$};

    \node[circle,draw=none] (dv) [left=4.1cm of dd] {\large $\vdots$};

    \node[circle,draw,minimum size =1cm] (a43) [left of=y4] {$1$};

    \node[circle,draw,minimum size =1cm] (a53) [left of=y5] {$0$};

    \node[regular polygon,regular polygon sides=4, draw,fill=gray!30,minimum size =1.25cm] (l33) [below left=0.01cm and 1.5cm of a43] {$ $};  
    \node[regular polygon,regular polygon sides=4, draw,fill=gray!30,minimum size =1.25cm] (l32) [left of=l33] {$ $};  
    \node[regular polygon,regular polygon sides=4, draw,fill=gray!30,minimum size =1.25cm] (l31) [left of=l32] {$ $};  
    \node[regular polygon,regular polygon sides=4, draw,fill=gray!30,minimum size =1.25cm] (l30) [left of=l31] {$ $};  

    \node[circle,draw=none] (dayd) [left= 1cm of l30] {$t = \Delta$};

    \path[->] (l33) edge node {} (a43);
    \path[->] (l33) edge node {} (a53);

    \path[->] (a43) edge node {} (y4);
    \path[->] (a53) edge node {} (y5);

    \end{tikzpicture}
   
    \caption{Schematic of nested daily trial design, on each day participants are randomly assigned to vaccine or no vaccine conditional on their history up to that day.}
    \label{fig:design1}
    \end{figure}

    \clearpage

    \begin{figure}[p]
        \centering
        \begin{tikzpicture}[> = stealth, shorten > = 1pt, auto, node distance = 1.5cm, thick]

            \node[circle,draw=none] (y0) {$Y$};
            \node[circle,draw=none] (y1) [below of=y0] {$Y$};
            \node[circle,draw=none] (y2) [below of=y1] {$Y$};
            \node[circle,draw=none] (y3) [below of=y2] {$Y$};
            \node[circle,draw=none] (d) [below of=y3] {\large $\vdots$};
            \node[circle,draw=none] (y4) [below of=d] {$Y$};
            \node[circle,draw=none] (y5) [below of=y4] {$Y$};
        
            \node[circle,draw,minimum size =1cm] (a03) [left of=y0] {$1$};
            \node[circle,draw,minimum size =1cm] (a02) [left of=a03] {$1$};
            \node[circle,draw,minimum size =1cm] (a01) [left of=a02] {$1$};
            \node[circle,draw,minimum size =1cm] (a0)  [left of=a01] {$1$};
        
            \node[circle,draw,minimum size =1cm] (a13) [left of=y1] {$0$};
            \node[circle,draw,minimum size =1cm] (a12) [left of=a13] {$0$};
            \node[circle,draw,minimum size =1cm] (a11) [left of=a12] {$0$};
            \node[circle,draw,minimum size =1cm] (a1)  [left of=a11] {$0$};
        
            \node[regular polygon,regular polygon sides=4, draw,fill=gray!30,minimum size =1.25cm] (l0) [below left=0.01cm and 1.5cm of a0] {$ $};  
            
            \node[circle,draw=none] (day0) [left= 1cm of l0] {$\delta = 0$};
        
            \path[->] (l0) edge node {} (a0);
            \path[->] (l0) edge node {} (a1);
            \path[->] (a0) edge node {} (a01);
            \path[->] (a1) edge node {} (a11);
        
            \path[->] (a11) edge node {} (a12);
            \path[->] (a12) edge node {} (a13);
            \path[->] (a13) edge node {} (y1);
        
            \path[->] (a01) edge node {} (a02);
            \path[->] (a02) edge node {} (a03);
            \path[->] (a03) edge node {} (y0);
        
            \node[circle,draw,minimum size =1cm] (a23) [left of=y2] {$1$};
            \node[circle,draw,minimum size =1cm] (a22) [left of=a23] {$1$};
            \node[circle,draw,minimum size =1cm] (a21) [left of=a22] {$1$};
            \node[circle, draw, dashed, minimum size =1cm] (a20) [left of=a21] {$?$}; 
    
            \node[circle,draw,minimum size =1cm] (a33) [left of=y3] {$0$};
            \node[circle,draw,minimum size =1cm] (a32) [left of=a33] {$0$};
            \node[circle,draw,minimum size =1cm] (a31) [left of=a32] {$0$};
            \node[circle, draw, dashed, minimum size =1cm] (a30) [left of=a31] {$?$};  
     
            \node[regular polygon,regular polygon sides=4, draw,fill=gray!30,minimum size =1.25cm] (l10) [below left=0.01cm and 1.5cm of a20] {$ $};  
        
            \node[circle,draw=none] (day1) [left= 1cm of l10] {$\delta = 1$};
        
    
            \path[->] (l10) edge node {} (a20);
            \path[->] (l10) edge node {} (a30);
    
            \path[->] (a20) edge node {} (a21);
            \path[->] (a21) edge node {} (a22);
            \path[->] (a22) edge node {} (a23);
            \path[->] (a23) edge node {} (y2);
    
            \path[->] (a30) edge node {} (a31);
            \path[->] (a31) edge node {} (a32);
            \path[->] (a32) edge node {} (a33);
            \path[->] (a33) edge node {} (y3);
        
            \node[circle,draw=none] (dd) [below of=a32] {\large $\ddots$};
        
            \node[circle,draw=none] (dv) [left=4.1cm of dd] {\large $\vdots$};
        
            \node[circle,draw,minimum size =1cm] (a43) [left of=y4] {$1$};
            \node[circle, draw, dashed, minimum size =1cm] (a42) [left of=a43] {$?$};  
            \node[circle, draw, dashed, minimum size =1cm] (a41) [left of=a42] {$?$};  
            \node[circle, draw, dashed, minimum size =1cm] (a40) [left of=a41] {$?$};  
    
            \node[circle,draw,minimum size =1cm] (a53) [left of=y5] {$0$};
            \node[circle, draw, dashed, minimum size =1cm] (a52) [left of=a53] {$?$};  
            \node[circle, draw, dashed, minimum size =1cm] (a51) [left of=a52] {$?$};  
            \node[circle, draw, dashed, minimum size =1cm] (a50) [left of=a51] {$?$};  
    
            % \node[regular polygon,regular polygon sides=4, draw,fill=gray!30,minimum size =1.25cm] (l33) [below left=0.01cm and 1.5cm of a43] {$ $};  
            % \node[regular polygon,regular polygon sides=4, draw,fill=gray!30,minimum size =1.25cm] (l32) [left of=l33] {$ $};  
            % \node[regular polygon,regular polygon sides=4, draw,fill=gray!30,minimum size =1.25cm] (l31) [left of=l32] {$ $};  
            \node[regular polygon,regular polygon sides=4, draw,fill=gray!30,minimum size =1.25cm] (l30) [below left=0.01cm and 1.5cm of a40] {$ $};  
        
            \node[circle,draw=none] (dayd) [left= 1cm of l30] {$\delta = \Delta - 1$};
        
            \path[->] (l30) edge node {} (a40);
            \path[->] (l30) edge node {} (a50);
        
            \path[->] (a40) edge node {} (a41);
            \path[->] (a50) edge node {} (a51);
            \path[->] (a41) edge node {} (a42);
            \path[->] (a51) edge node {} (a52);
            \path[->] (a42) edge node {} (a43);
            \path[->] (a52) edge node {} (a53);
            \path[->] (a43) edge node {} (y4);
            \path[->] (a53) edge node {} (y5);
    
            \end{tikzpicture}
            \caption{Schematic of trials with different grace periods, participants are randomized to a strategy starting on day zero and given a fixed length time window in which they can initiate and then sustain thereafter.}
            \label{fig:design2}
    \end{figure}

    \clearpage 
    \subsection{Example emulation from hypothetical dataset}

    
    \subsubsection{Grace period}

    \begin{table}[p]
        \small
        \centering
        \caption{Enrollment of six hypothetical individuals in trial with a 3-day grace period.\label{tab:example1}}
        \begin{tabular}{cC{1in}C{1in}C{1.1in}C{1.2in}C{1in}}
        \toprule
        \makecell[c]{ID} & \makecell[c]{Vaccination \\ (1: yes, 0: no)} & \makecell[c]{Day of \\ vaccination} & \makecell[c]{Clinical disease \\ (1: yes, 0: no)} & \makecell[c]{Day of \\ symptom onset} & \makecell[c]{No. of regimes \\ followed} \\
        \midrule
            1 & 1 & 2 & 0 & - & 3 (0-2) \\
            2 & 1 & 4 & 1 & 8 & 3 (0-2) \\
            3 & 1 & 7 & 1 & 3 & 4 (0-3) \\
            4 & 0 & - & 0 & - & 14 (0-13) \\
            5 & 0 & - & 1 & 5 & 6 (0-5) \\
            6 & 0 & - & 1 & 9 & 10 (0-9) \\
        \bottomrule
        \end{tabular}
        \end{table}

        
    \subsubsection{Nested daily trials}

    \begin{table}[p]
        \small
        \centering
        \caption{Enrollment of six hypothetical individuals in daily nested trials for 14-day vaccination window based on observed data.\label{tab:example2}}
        \begin{tabular}{cC{1in}C{1in}C{1.1in}C{1.2in}C{1in}}
        \toprule
        \makecell[c]{ID} & \makecell[c]{Vaccination \\ (1: yes, 0: no)} & \makecell[c]{Day of \\ vaccination} & \makecell[c]{Clinical disease \\ (1: yes, 0: no)} & \makecell[c]{Day of \\ symptom onset} & \makecell[c]{No. of trials \\ enrolled} \\
        \midrule
            1 & 1 & 2 & 0 & - & 3 (0-2) \\
            2 & 1 & 4 & 1 & 8 & 5 (0-4) \\
            3 & 1 & 7 & 1 & 3 & 4 (0-3) \\
            4 & 0 & - & 0 & - & 14 (0-13) \\
            5 & 0 & - & 1 & 5 & 6 (0-5) \\
            6 & 0 & - & 1 & 9 & 10 (0-9) \\
        \bottomrule
        \end{tabular}
        \end{table}
    
    
\end{appendix}


\end{document}
%TC:endignore

\onehalfspacing

\end{document}